\documentclass[]{article}
\usepackage{lmodern}
\usepackage{amssymb,amsmath}
\usepackage{ifxetex,ifluatex}
\usepackage{fixltx2e} % provides \textsubscript
\ifnum 0\ifxetex 1\fi\ifluatex 1\fi=0 % if pdftex
  \usepackage[T1]{fontenc}
  \usepackage[utf8]{inputenc}
\else % if luatex or xelatex
  \ifxetex
    \usepackage{mathspec}
  \else
    \usepackage{fontspec}
  \fi
  \defaultfontfeatures{Ligatures=TeX,Scale=MatchLowercase}
\fi
% use upquote if available, for straight quotes in verbatim environments
\IfFileExists{upquote.sty}{\usepackage{upquote}}{}
% use microtype if available
\IfFileExists{microtype.sty}{%
\usepackage{microtype}
\UseMicrotypeSet[protrusion]{basicmath} % disable protrusion for tt fonts
}{}
\usepackage[margin=1in]{geometry}
\usepackage{hyperref}
\hypersetup{unicode=true,
            pdftitle={Glycolysis},
            pdfborder={0 0 0},
            breaklinks=true}
\urlstyle{same}  % don't use monospace font for urls
\usepackage{graphicx,grffile}
\makeatletter
\def\maxwidth{\ifdim\Gin@nat@width>\linewidth\linewidth\else\Gin@nat@width\fi}
\def\maxheight{\ifdim\Gin@nat@height>\textheight\textheight\else\Gin@nat@height\fi}
\makeatother
% Scale images if necessary, so that they will not overflow the page
% margins by default, and it is still possible to overwrite the defaults
% using explicit options in \includegraphics[width, height, ...]{}
\setkeys{Gin}{width=\maxwidth,height=\maxheight,keepaspectratio}
\IfFileExists{parskip.sty}{%
\usepackage{parskip}
}{% else
\setlength{\parindent}{0pt}
\setlength{\parskip}{6pt plus 2pt minus 1pt}
}
\setlength{\emergencystretch}{3em}  % prevent overfull lines
\providecommand{\tightlist}{%
  \setlength{\itemsep}{0pt}\setlength{\parskip}{0pt}}
\setcounter{secnumdepth}{0}
% Redefines (sub)paragraphs to behave more like sections
\ifx\paragraph\undefined\else
\let\oldparagraph\paragraph
\renewcommand{\paragraph}[1]{\oldparagraph{#1}\mbox{}}
\fi
\ifx\subparagraph\undefined\else
\let\oldsubparagraph\subparagraph
\renewcommand{\subparagraph}[1]{\oldsubparagraph{#1}\mbox{}}
\fi

%%% Use protect on footnotes to avoid problems with footnotes in titles
\let\rmarkdownfootnote\footnote%
\def\footnote{\protect\rmarkdownfootnote}

%%% Change title format to be more compact
\usepackage{titling}

% Create subtitle command for use in maketitle
\newcommand{\subtitle}[1]{
  \posttitle{
    \begin{center}\large#1\end{center}
    }
}

\setlength{\droptitle}{-2em}

  \title{Glycolysis}
    \pretitle{\vspace{\droptitle}\centering\huge}
  \posttitle{\par}
    \author{}
    \preauthor{}\postauthor{}
    \date{}
    \predate{}\postdate{}
  

\begin{document}
\maketitle

\hypertarget{imports}{%
\section{Imports}\label{imports}}

\hypertarget{glycolysis}{%
\section{Glycolysis}\label{glycolysis}}

Glycolysis is a metabolic pathway involving the stepwise degradation of
glucose

\hypertarget{background}{%
\subsection{Background}\label{background}}

Glycolysis is a very ancient process, which arose relatively early on in
evolutionary history. As a result of its age, today almost all living
organisms undergo some form of glycolysis. Glycolysis is an anaerobic
process, as it arose before oxygen formed a large substituent of the
earths atmosphere. Even in aerobic organism glycolysis still remains
anaerobic, although the products of glycolysis are then fed into other
aerobic pathways.

\hypertarget{process}{%
\subsection{Process}\label{process}}

Glycolysis involves the break down of glucose molecule into pyruvate
through ten enzyme catalysed steps. These ten steps can be generally
grouped into two phases, the initially priming phase in which glucose is
transformed into a less stable form, a process which requires energy
provided by coupling with ATP hydrolysis, and the second energy release
phase in which new ATP is generated. For each one molecule of glucose
metabolised, two molecules of pyruvate are produced. The balanced
equation for the overall reaction is as follows: \newline 
\(Glucose +2Pi+4NAD^++2ADP \rightarrow 4NADN+2H_2O+22TP+4H^++2Acetyl-CoA+2CO_2+2CoASH\)

\hypertarget{phase-1-priming}{%
\subsubsection{Phase \#1 (Priming)}\label{phase-1-priming}}

Phase one includes the first 4 reactions of glycolysis, in which,
overall, Glyceraldehyde-3-phosphate is produced by the splitting of
glucose. The balanced equation for this phase is as follows: \newline 

\hypertarget{energy-requirements}{%
\subsubsection{Energy requirements}\label{energy-requirements}}

2 ATP molecules are used up in this phase.

\hypertarget{reactions}{%
\subsubsection{Reactions}\label{reactions}}

\hypertarget{reaction-1}{%
\paragraph{Reaction \#1}\label{reaction-1}}

Glucose is phosphorylated by hexokinase or glucokinase to form
glucose-6-phosphate, coupled with the hydrolysis of one ATP molecule.
The balanced equation for this reaction is as follows.

\(\text{Glucose} + ATP \rightarrow \text{glucose-6-phosphate} + ADP + H^+\)
\(\quad \quad\) \(\Delta G^\circ = -16.7 kJ\cdot ml^{-1}\)

\begin{quote}
NOTE: Extracellular glucose will move into the cell as glucose is
converted from glucose to glucose 6 phosphate, by the law of mass
action.
\end{quote}

\begin{quote}
NOTE: Reaction one is often referred to as the first priming reaction.
\end{quote}

\hypertarget{energetics}{%
\subparagraph{Energetics}\label{energetics}}

The reaction itself is not thermodynamically favorable, and hence must
be coupled with ATP hydrolysis to achieve spontaneity.

\hypertarget{enzymes}{%
\subparagraph{Enzymes}\label{enzymes}}

The reaction can be catalysed by glucokinase, or hexokinase. Both
glucokinase and hexokinase require magnesium ions to function.
Glucokinase only functions at high glucose levels whereas hexokinase
binds glucose at low glucose concentrations, hence glucokinase is only
active after consumption of a high glucose meal.

Regulation.

Hexokinase is one of the regulated enzymes in glycolysis, the reaction
is targeted as a regulation point due to its large negative free energy
change.

\hypertarget{reaction-2}{%
\paragraph{Reaction \#2}\label{reaction-2}}

Glucose-6 Phosphate is converted into fructose 6 phosphate by
phosphoglucoisomerase.

\hypertarget{enzyme}{%
\subparagraph{Enzyme}\label{enzyme}}

Phosphoglucoisomerase is used for the reaction. This enzyme requires
magnesium ions to function.

\hypertarget{reaction-3}{%
\paragraph{Reaction \#3}\label{reaction-3}}

Fructose-6-phosphate is phosphorylated into fructose-1,6-bisphosphate by
phosphofructokinase. The reaction is natively endothermic so must
coupled with ATP Hydrolysis to occur. The balanced reaction is given
as:\newline

\(\text{Fructose-6-phosphate}+ATP\rightarrow \text{fructose-1,6,-bisphosphate}+ ATP \quad\quad \Delta G^\circ \prime=14.2kJ\cdot mol^{-1}\)

\begin{quote}
NOTE: Reaction 3 is often referred to as the second primary reaction.
\end{quote}

\hypertarget{enzyme-1}{%
\subparagraph{Enzyme}\label{enzyme-1}}

The reaction is catalysed by phosphofructokinase.

Regulation

The phosphofructokinase reaction controls the rate of glycolysis.
Intermediates of the citric acid cycle form allosteric inhibitor's which
act on phosphofructokinase. \#\#\# Reaction \#4 Fructose biphosphate
aldose, producs two 3 carbo intermediate.

\hypertarget{phase-2}{%
\subsubsection{Phase \#2}\label{phase-2}}

Phase two includes the last 6 reactions of glycolysis in which,
overall,glyceraldehyde 3 phosphate is converted to pyruvate.

\hypertarget{reaction-5}{%
\subsubsection{Reaction \#5}\label{reaction-5}}

Triose phosphare isomerase compelets the first phase of glycolysis by
catalysing an ismoerisation reaction.

\hypertarget{phase-2-1}{%
\subsubsection{Phase 2}\label{phase-2-1}}

\hypertarget{reaction-6}{%
\subsubsection{Reaction \#6}\label{reaction-6}}

Glyceraldehyde-3-phosphate dehydrogenase produces a high energy
intermediate in an oxidation reaction.

In the absence of oxygen reaction 6 cannot occur if there is no
\(NAD^+\) the recycling of \(NAD^+\) via the reduction of pyruvate.
\#\#\# reaction \#7

\hypertarget{spontaneity}{%
\paragraph{Spontaneity}\label{spontaneity}}

Very spontaneous coupling with reaction six, drives reaction six

\hypertarget{reaction}{%
\paragraph{Reaction}\label{reaction}}

\(1,3-BPG+ADP \rightarrow 3 phosphoglycerate+ATP\)
\(\Delta G^\circ ' -18.9kJ.mol^{-1}\)

\hypertarget{enzyme-2}{%
\paragraph{Enzyme}\label{enzyme-2}}

Phosphoglycerate Kinase

\hypertarget{cofactors}{%
\paragraph{Cofactors}\label{cofactors}}

\hypertarget{atp-production}{%
\paragraph{ATP Production}\label{atp-production}}

regenerates ATP used earlier in Glycolysis.

Phosphoglycerate kinase (\(Mg^{2+}\)) as in the back reaction the
phosphoglycerate is phosphorylated. or it could be called
1,3-Bisphosphate phosphatase.

this reaction has a large negative \(\Delta G\) and hence is a site of
regulation.

first substrate level phosphrylation occurs, ie the phosphate group
derrives dirrectly from the phosphate, in the ETC ATP is formed from ATP
by substrate level phosphorylation.

Break even reaction, energy lost is equal to energy gained

Reaction \#7

\(1,3-BPG+ADP \rightarrow 3 phosphoglycerate+ATP\) \(\Delta G^\circ '\)
-18.9kJ.mol\^{}\{-1\}. Phosphoglycerate kinase (\(MG^{2+}\)) as in the
back reaction the phosphoglycerate is phosphorylated. or it could be
called 1,3-Bisphosphate phosphatase.

this reaction has a large negative \(\Delta G\) and hence is a site of
regulation.

first substrate level phosphrylation occurs, ie the phosphate group
derrives dirrectly from the phosphate, in the ETC ATP is formed from ATP
by substrate level phosphorylation.

Break even reaction, energy lost is equal to energy gained

\hypertarget{reaction-8}{%
\subsubsection{Reaction \#8}\label{reaction-8}}

\hypertarget{phosphoglycerate-mutase-mutases-move-only-one-functional-group.-this-reaction-prepares-the-substrate-for-the-following-reaction-involving-the-synthesis-of-atp-ie-priming-the-phosphate-group-for-removal.}{%
\paragraph{Phosphoglycerate mutase (mutases move only one functional
group). this reaction prepares the substrate for the following reaction
involving the synthesis of ATP, ie priming the phosphate group for
removal.}\label{phosphoglycerate-mutase-mutases-move-only-one-functional-group.-this-reaction-prepares-the-substrate-for-the-following-reaction-involving-the-synthesis-of-atp-ie-priming-the-phosphate-group-for-removal.}}

\hypertarget{reaction-9}{%
\subsubsection{Reaction \#9}\label{reaction-9}}

\hypertarget{enzyme-3}{%
\paragraph{Enzyme}\label{enzyme-3}}

Enolase.

\hypertarget{reaction-4}{%
\paragraph{Reaction}\label{reaction-4}}

Dehydration reaction

\(2 phosphoglycerate \rightarrow phosphoenolpyruvate\)

As much more instable in the enol form so it is a muich higher energy
molecule (it has not gained energy, only instability). enols tautomerise
to ketones.

produces PEP via dehydrogenase (?)

\hypertarget{reaction-10}{%
\subsubsection{Reaction \#10}\label{reaction-10}}

\hypertarget{enzyme-4}{%
\paragraph{Enzyme}\label{enzyme-4}}

pyruvate kinase (phosphoenolpyruvate phosphatase)

transfer of a phosphoryl group from PEP to ADP to generate ATP and
Pyruvate, enol tautomer converted to more favourable keto tautomer,

tautomers are isomers which can readily intraconvert between each other.

\hypertarget{reaction-7}{%
\paragraph{Reaction}\label{reaction-7}}

\(Phosphoenolpyruvate \rightarrow pyruvate\) pyruvate Kinase produces
pyruvate and mro ATP

\hypertarget{kinase-enzymes.}{%
\subsubsection{Kinase enzymes.}\label{kinase-enzymes.}}

transfer a phosphate group.

\hypertarget{phosphatase}{%
\subsubsection{phosphatase}\label{phosphatase}}

a dephosphorylating enzyme.

\hypertarget{location}{%
\subsection{Location}\label{location}}

Glycolysis occurs in the cytoplasm.

\hypertarget{biological-significance}{%
\subsection{Biological significance}\label{biological-significance}}

Glycolysis is a particularly important process as in many cases glucose
is the only source of metabolic energy. Cells of the brain kidney
contractual skeletal muscles, erythrocytes, and sperm cell, are all
solely reliant on glycolysis for energy.

\hypertarget{products}{%
\subsection{Products}\label{products}}

Pyruvate is a versatile metabolite, depending on the availability of
oxygen, pyruvate can undergo further oxidation to form acetyl-CoA and
carbon dioxide. If oxygen is not present then in animal cells the
pyruvate will be converted to lactic acid (lactate), and in yeast cells
it will undergo alcoholic fermentation to form ethanol.

\hypertarget{stage-2}{%
\paragraph{Stage 2}\label{stage-2}}

amino acids to alpha keto acids, and then pyruvate or acetylp co A
glycerol and glucose to pyruvate. and fatty acids to acetyl CoA.
pyruvate can be further brocked down into acetyl-coA (the common end
product of metabolism

final waste products are carbon dioxide and water.

regulation of glycolysis COPY IMAGE. no significant trend in free
energyies released, however under physiological conditions all most run
close to eqilibrium,

\hypertarget{appendix}{%
\section{Appendix}\label{appendix}}


\end{document}
